\documentclass[10pt,twoside]{book}

% Packages for dictionary layout
\usepackage[utf8]{inputenc}
\usepackage[T1]{fontenc}
\usepackage{geometry}
\usepackage{fancyhdr}
\usepackage{multicol}
\usepackage{xcolor}
\usepackage{titlesec}
\usepackage{enumitem}
\usepackage{fontspec}
\usepackage{microtype}
\usepackage{ragged2e}
\usepackage{hyphenat}
\usepackage{xstring}
\usepackage{hyperref}

% Hyperref configuration
\hypersetup{
    colorlinks=true,
    linkcolor=black,
    filecolor=magenta,      
    urlcolor=cyan,
    pdftitle={Shona-English Dictionary},
    pdfauthor={Duramazwi},
    pdfsubject={Shona Language Dictionary},
    pdfkeywords={Shona, English, Dictionary, Language},
    bookmarks=true,
    bookmarksopen=true,
    bookmarksnumbered=false
}

% Page geometry - compact dictionary style
\geometry{
    paperwidth=6in,
    paperheight=9in,
    margin=0.75in,
    top=0.6in,
    bottom=0.6in,
    headheight=14pt,
    headsep=12pt,
    footskip=20pt
}

% Font settings
\setmainfont{Times New Roman}
\setsansfont{Arial}

% Colors
\definecolor{headwordcolor}{RGB}{0,50,100}
\definecolor{partofpeechcolor}{RGB}{100,100,100}
\definecolor{examplecolor}{RGB}{120,120,120}
\definecolor{englishexamplecolor}{RGB}{80,80,80}
\definecolor{prefixcolor}{RGB}{34,197,94}
\definecolor{verbcolor}{RGB}{34,197,94}
\definecolor{nouncolor}{RGB}{59,130,246}
\definecolor{adverbcolor}{RGB}{234,179,8}

% Header/Footer setup
\pagestyle{fancy}
\fancyhf{}
\fancyhead[LE]{\textbf{\thepage} \hfill \textsc{\leftmark}}
\fancyhead[RO]{\textsc{\rightmark} \hfill \textbf{\thepage}}
\renewcommand{\headrulewidth}{0.4pt}

% Chapter and section formatting
\titleformat{\chapter}[display]
{\normalfont\huge\bfseries\centering}
{\chaptertitlename\ \thechapter}{20pt}{\Huge}

% Ensure headers appear on chapter pages
\fancypagestyle{plain}{%
  \fancyhf{}%
  \fancyhead[LE]{\textbf{\thepage} \hfill \textsc{\leftmark}}%
  \fancyhead[RO]{\textsc{\rightmark} \hfill \textbf{\thepage}}%
  \renewcommand{\headrulewidth}{0.4pt}%
}

\titleformat{\section}[block]
{\normalfont\Large\bfseries\centering}
{\thesection}{1em}{}

% Custom command for dictionary letter headers (with page breaks)
\newcommand{\letterheader}[1]{%
    \clearpage%                              % Force page break
    \phantomsection%                         % Create anchor for hyperref
    \vspace{20pt plus 4pt minus 2pt}%       % Space from top
    \begin{center}%
        {\normalfont\LARGE\bfseries #1}%     % Large bold letter
    \end{center}%
    \vspace{10pt plus 2pt minus 2pt}%       % Space before content
    \markboth{#1}{#1}%                      % Header marks
    \label{chap:#1}%                        % Create label
    \addcontentsline{toc}{chapter}{#1}%     % Add to TOC as chapter
}

% First letter header (no page break)
\newcommand{\firstletterheader}[1]{%
    \phantomsection%                         % Create anchor for hyperref
    \vspace{20pt plus 4pt minus 2pt}%       % Space from top
    \begin{center}%
        {\normalfont\LARGE\bfseries #1}%     % Large bold letter
    \end{center}%
    \vspace{10pt plus 2pt minus 2pt}%       % Space before content
    \markboth{#1}{#1}%                      % Header marks
    \label{chap:#1}%                        % Create label
    \addcontentsline{toc}{chapter}{#1}%     % Add to TOC as chapter
}

% Custom commands for dictionary entries
\newcommand{\headword}[1]{%
    \textbf{\color{headwordcolor}\large #1}%
}

% Command for verb headwords with colored ku- prefix
\newcommand{\verbheadword}[2]{%
    \textbf{\large\color{prefixcolor}#1\color{headwordcolor}#2}%
}

\newcommand{\pronunciation}[1]{%
    \textit{[#1]}%
}

% Specific part of speech commands with colors
\newcommand{\verbpos}{\textit{\color{verbcolor}verb}}
\newcommand{\nounpos}{\textit{\color{nouncolor}noun}}
\newcommand{\adverbpos}{\textit{\color{adverbcolor}adverb}}
\newcommand{\adjectivepos}{\textit{\color{adverbcolor}adjective}}

% Generic part of speech command (fallback)
\newcommand{\partofpeech}[1]{%
    \textit{\color{partofpeechcolor}#1}%
}

\newcommand{\definition}[1]{%
    #1%
}

\newcommand{\example}[2]{%
    \textit{\color{examplecolor}#1} \textit{#2}%
}

\newcommand{\crossref}[1]{%
    \textsc{see} \textbf{#1}%
}

% Dictionary entry environment (deprecated - now handled in generator)
% \newenvironment{dictentry}[1]{%
%     \par\noindent\headword{#1}\space%
% }{%
%     \par\vspace{4pt}%
% }

% Compact list for multiple definitions
\newlist{deflist}{enumerate}{1}
\setlist[deflist]{label=\textbf{\arabic*.},leftmargin=1em,itemsep=0pt,parsep=0pt,topsep=2pt}

% Page breaking settings to prevent orphans and widows
\widowpenalty=10000
\clubpenalty=10000
\raggedbottom
\setlength{\columnsep}{1em}

% Ragged right settings for better text flow
\setlength{\RaggedRightParindent}{0pt}
\setlength{\RaggedRightRightskip}{0pt plus 2em}

% Column balancing settings
\setlength{\columnseprule}{0pt}
% \raggedcolumns  % Commented out - may not be available in all distributions

% Prevent breaking within dictionary entries
\newenvironment{dictentryblock}{%
    \begin{samepage}%
}{%
    \end{samepage}%
}

% Title page information
\title{%
    {\Huge\textbf{SHONA-ENGLISH}}\\[0.5cm]
    {\Large\textbf{DICTIONARY}}\\[1cm]
    {\large Duramazwi reChiShona neChirungu}
}

\author{%
    Compiled from Traditional Sources\\
    and Modern Usage\\[0.5cm]
    {\small Version 1.0}
}

\date{\today}

\begin{document}

% Front matter
\frontmatter
\maketitle

% Copyright page
\newpage
\thispagestyle{empty}
\vspace*{\fill}
\begin{center}
\textbf{Shona-English Dictionary}\\[0.5cm]
\textcopyright\ \the\year\ shonadictionary.com\\[0.5cm]

This dictionary is compiled from traditional Shona sources\\
and modern community contributions.\\[1cm]

Published under Creative Commons License\\
Attribution-ShareAlike 4.0 International\\[0.5cm]

First Edition\\
\today
\end{center}
\vspace*{\fill}

% Table of contents
\tableofcontents
\newpage

% Preface
\chapter*{Preface}
\addcontentsline{toc}{chapter}{Preface}

This dictionary represents a comprehensive collection of Shona words and their English translations, compiled from both traditional sources and modern usage. The Shona language, spoken by over 10 million people primarily in Zimbabwe, is a rich and vibrant language with a deep cultural heritage.

Our goal is to preserve and promote the Shona language by making it accessible to both native speakers and learners worldwide. This dictionary includes:

\begin{itemize}
\item Over 1,000 carefully selected Shona words
\item Clear English definitions and translations
\item Example sentences in both languages
\item Pronunciation guides where applicable
\item Cross-references to related terms
\end{itemize}

The entries in this dictionary follow standard Shona orthography and include words from various domains of life, from traditional concepts to modern terminology.

\vspace{1cm}
\textit{Tinoda kuchengetedza mutauro wechiShona nokuitenderedza pasi rose. Duramazwi iri rine mazwi anobva muupenyu hwemazuva ano uye hwekare.}

% Introduction
\chapter*{Introduction}
\addcontentsline{toc}{chapter}{Introduction}

\section*{About the Shona Language}

Shona is a Bantu language spoken primarily in Zimbabwe, with smaller communities in Mozambique, Botswana, and Zambia. It is one of the official languages of Zimbabwe, alongside English and Ndebele.

\section*{Pronunciation Guide}

Shona uses a relatively simple phonetic system. Most letters are pronounced as they appear, with the following key points:

\begin{itemize}
\item \textbf{a} - as in "father"
\item \textbf{e} - as in "bed"
\item \textbf{i} - as in "machine"
\item \textbf{o} - as in "more"
\item \textbf{u} - as in "moon"
\end{itemize}

Special combinations:
\begin{itemize}
\item \textbf{bh} - aspirated b
\item \textbf{ch} - as in "church"
\item \textbf{dh} - aspirated d
\item \textbf{gh} - aspirated g
\item \textbf{kh} - aspirated k
\item \textbf{ph} - aspirated p
\item \textbf{th} - aspirated t
\item \textbf{zh} - as in "measure"
\end{itemize}

\section*{How to Use This Dictionary}

Each entry follows this format:
\begin{itemize}
\item \textbf{Headword} - the main Shona word
\item \textit{Part of speech} - noun, verb, adjective, etc.
\item Definition - clear English explanation
\item \textit{Example} - usage in context with translation
\end{itemize}

% Main dictionary content
\mainmatter

%SAMPLE_CHAPTERS_PLACEHOLDER%

% Back matter
\backmatter

% Appendices
\appendix

\chapter{Common Phrases}
\begin{multicols}{2}
\RaggedRight

\textbf{Greetings}
\begin{itemize}[leftmargin=*]
\item Mangwanani - Good morning
\item Masikati - Good afternoon  
\item Manheru - Good evening
\item Usiku hwakanaka - Good night
\end{itemize}

\textbf{Basic Expressions}
\begin{itemize}[leftmargin=*]
\item Ndatenda - Thank you
\item Pamusoroi - Excuse me/Sorry
\item Ndinokuda - I love you
\item Ndine nzara - I am hungry
\item Ndine nyota - I am thirsty
\end{itemize}

\textbf{Questions}
\begin{itemize}[leftmargin=*]
\item Makadii? - How are you?
\item Zita renyu ndiani? - What is your name?
\item Munobva kupi? - Where are you from?
\item Nguva yakadii? - What time is it?
\end{itemize}

\end{multicols}

\chapter{Numbers}
\begin{multicols}{2}
\RaggedRight

\textbf{Cardinal Numbers}
\begin{itemize}[leftmargin=*]
\item motsi - one
\item piri - two
\item tatu - three
\item ina - four
\item shanu - five
\item tanhatu - six
\item nomwe - seven
\item sere - eight
\item pfumbamwe - nine
\item gumi - ten
\end{itemize}

\end{multicols}

% Bibliography/Sources
\chapter*{Sources and References}
\addcontentsline{toc}{chapter}{Sources and References}

This dictionary was compiled from the following sources:

\begin{itemize}
\item Traditional Shona oral literature and proverbs
\item Hannan, M. \textit{Standard Shona Dictionary}. Harare: Literature Bureau.
\item Dale, D. \textit{Duramazwi: A Shona-English Dictionary}. Gweru: Mambo Press.
\item Community contributions from shonadictionary.com
\item Modern usage examples from contemporary Shona speakers
\end{itemize}

\vspace{1cm}
\textit{Special thanks to the Shona-speaking community worldwide for their contributions to preserving and promoting this beautiful language.}

\end{document}
